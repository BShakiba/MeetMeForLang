%%%%%%%%%%%%%%%%%%%%%%%%%%%%%%%%%%%%%%%%%
% Stylish Colored Title Page 
% LaTeX Template
% Version 1.0 (27/12/12)
%
% This template has been downloaded from:
% http://www.LaTeXTemplates.com
%
% Original author:
% Peter Wilson (herries.press@earthlink.net)
%
% License:
% CC BY-NC-SA 3.0 (http://creativecommons.org/licenses/by-nc-sa/3.0/)
% 
% Instructions for using this template:
% This title page compiles as is. If you wish to include this title page in 
% another document, you will need to copy everything before 
% \begin{document} into the preamble of your document. The title page is
% then included using \titleBC within your document.
%
%%%%%%%%%%%%%%%%%%%%%%%%%%%%%%%%%%%%%%%%%

%----------------------------------------------------------------------------------------
%	PACKAGES AND OTHER DOCUMENT CONFIGURATIONS
%----------------------------------------------------------------------------------------

\documentclass{article}

\usepackage[svgnames]{xcolor} % Required to specify font color

\newcommand*{\plogo}{\fbox{Drexel University}} % Generic publisher logo

\usepackage{graphicx} % Required for box manipulation

%----------------------------------------------------------------------------------------
%	TITLE PAGE
%----------------------------------------------------------------------------------------

\newcommand*{\rotrt}[1]{\rotatebox{90}{#1}} % Command to rotate right 90 degrees
\newcommand*{\rotlft}[1]{\rotatebox{-90}{#1}} % Command to rotate left 90 degrees

\newcommand*{\titleBC}{\begingroup % Create the command for including the title page in the document
\centering % Center all text

\def\CP{\textit{\Huge Meet Me For Language}} % Title

\settowidth{\unitlength}{\CP} % Set the width of the curly brackets to the width of the title
{\color{LightGoldenrod}\resizebox*{\unitlength}{\baselineskip}{\rotrt{$\}$}}} \\[\baselineskip] % Print top curly bracket
\textcolor{Sienna}{\CP} \\[\baselineskip] % Print title
{\color{RosyBrown}\Large Profiles for Socializing via Language Learning} \\ % Tagline or further description
{\color{LightGoldenrod}\resizebox*{\unitlength}{\baselineskip}{\rotlft{$\}$}}} % Print bottom curly bracket

\vfill % Whitespace between the title and the author name

{\Large\textbf{Priyasmita Bagchi}}\\ % Author name
{\Large\textbf{Bahareh Shakibajahromi}}\\ % Author name
{\Large\textbf{Mike Bok}}\\ % Author name
{\Large\textbf{Ismail Kuru}}\\ % Author name

\vfill % Whitespace between the author name and the publisher logo

\plogo\\[0.5\baselineskip] % Publisher logo
2016 % Year published

\endgroup}

%----------------------------------------------------------------------------------------
%	BLANK DOCUMENT
%----------------------------------------------------------------------------------------

\usepackage{longtable}
\usepackage{float}
\usepackage[section]{placeins}
\usepackage[toc,page]{appendix}
\usepackage{graphicx}
\usepackage[export]{adjustbox}
\usepackage{pdfpages}
\usepackage{amssymb}
\usepackage[utf8x]{inputenc}
\usepackage{amsmath}
\usepackage{amsthm}
\usepackage{listings}
\usepackage{stmaryrd}
\usepackage{graphics,graphicx}
%\usepackage[colorinlistoftodos]{todonotes}
\usepackage{import}
\usepackage{mathpartir}
\usepackage{aliascnt}
\usepackage{array}
\usepackage{comment}
\usepackage{indentfirst}
\usepackage{tikz}
\usetikzlibrary{matrix}
\usepackage{tikz}
\usepackage{environ}
\usetikzlibrary{calc, shapes, backgrounds}
\usepackage{verbatim}
\usepackage{pgf}
\usepackage{tikz}
\usepackage{tikz-qtree}
\usepackage{pgfplots}
\usepackage{caption,capt-of}
\usepackage{subcaption}
\usepackage{color}
\usepackage{xcolor}
\newtheorem{theorem}{Theorem}
\newtheorem{corollary}{Corollary}[theorem]
\newtheorem{Lemma}[theorem]{Lemma}
\theoremstyle{definition}
\newtheorem{definition}{Definition}[section]
\theoremstyle{remark}
\newtheorem*{remark}{Remark}
\usepackage{color}
\newcommand{\hilight}[1]{\colorbox{yellow}{#1}}
\usepackage{multicol}
\makeatletter

\usepackage{tcolorbox}
\usepackage{mdframed}
%\input{colin_macros.tex}

\mdfdefinestyle{MyFrame}{%
    linecolor=blue,
    outerlinewidth=2pt,
    roundcorner=20pt,
    innertopmargin=\baselineskip,
    innerbottommargin=\baselineskip,
    innerrightmargin=20pt,
    innerleftmargin=20pt,
    backgroundcolor=gray!50!white}

\pgfplotsset{}
\usetikzlibrary{shapes.geometric,arrows,fit,matrix,positioning}
\tikzset
{
    treenode/.style = {circle, draw=black, align=center, minimum size=0.1cm},
    subtree/.style  = {isosceles triangle, draw=black, align=center, minimum height=0.5cm, minimum width=0.5cm, shape border rotate=90, anchor=north}
}

\usetikzlibrary{graphs} % LaTeX
\usetikzlibrary{arrows,automata}
\usepackage{hyperref}

\definecolor{sh_comment}{rgb}{0.12, 0.38, 0.18 } %adjusted, in Eclipse: {0.25, 0.42, 0.30 } = #3F6A4D
\definecolor{sh_keyword}{rgb}{0.37, 0.08, 0.25}  % #5F1441
\definecolor{sh_string}{rgb}{0.06, 0.10, 0.98} % #101AF9


\def\lstsmallmath{\leavevmode\ifmmode \scriptstyle \else  \fi}
\def\lstsmallmathend{\leavevmode\ifmmode  \else  \fi}

\lstset {
 rulesepcolor=\color{black},
 showspaces=false,showtabs=false,tabsize=2,
 captionpos=b,
 xleftmargin=0.7cm, xrightmargin=0.5cm,
 lineskip=-0.3em,
 linewidth=\linewidth,
 language=C++,
 basicstyle=\ttfamily,
 keywordstyle=\color{blue}\ttfamily,
 stringstyle=\color{red}\ttfamily,
 commentstyle=\color{sh_comment}\ttfamily,
 morecomment=[l][\color{magenta}]{\#},
 escapeinside={<@}{@>}
}
\usepackage{listings}
\usepackage{color}
 
\definecolor{codegreen}{rgb}{0,0.6,0}
\definecolor{codegray}{rgb}{0.5,0.5,0.5}
\definecolor{codepurple}{rgb}{0.58,0,0.82}
\definecolor{backcolour}{rgb}{0.95,0.95,0.92}
 
\lstdefinestyle{mystyle}{
 %   backgroundcolor=\color{backcolour},   
    commentstyle=\color{codegreen},
    keywordstyle=\color{magenta},
    numberstyle=\tiny\color{codegray},
    stringstyle=\color{codepurple},
    basicstyle=\footnotesize,
    linewidth=\linewidth,
    breakatwhitespace=false,         
    breaklines=true,                 
    captionpos=b,                    
    keepspaces=true,                 
    numbers=left,                    
    numbersep=5pt,                  
    showspaces=false,                
    showstringspaces=false,
    showtabs=false,                  
    tabsize=2
}
 
\lstset{style=mystyle}
\usepackage{amsmath}

\begin{document}
\pagestyle{empty} % Removes page numbers

\titleBC % This command includes the title page
\newpage
\section{Motivation}
This project aims providing a platform for users to learn and improve their non-native languages via socializing with other  people within a university. It is a profile based platform which enables users to create their profiles and get matched to a language learning event. This platform provides not only a pure learning activity but also getting socialized via meet-ups of users in created events.
\section{Functionality}
\subsection{Creating a Profile}
A user creates a profile to get involved in the platform.
 \subsubsection{Indentification}Since this is a platform for university community, their employee, student identification number will be their user identification number. 
 \subsubsection{Lecturing Languages In Native} Each user have languages that they are interested in teaching.  
 \subsubsection{Learning Languages}Each user have languages that they are interested in learning.
\subsection{Qualification Aspect} Each language stated in the profile, for teaching and learning, needs to be attached with a qualification, native, advanced, intermediate, basic etc.  
 \subsection{Type of Lecturing Method} A user may want to meet with a group of people to teach. Maximum size of group needs to be identified. Besides, she may prefer to meet one-to-one for lecturing.
 \subsection{Type of Learning Method}  A user may want to meet with a group of people to learn. Maximum size of group needs to be identified. Besides, she may prefer to meet one-to-one for learning.
 \subsection{Type of Events to Involve} A user states the type of events that are interested in attending. For example, a user may not want to attend any outdoor event etc.
 \subsection{Availability for Events} Each user states her available time intervals for the events.
 \subsection{Events Attended} A user has list of events that she attended to lecture or learn. 
 \subsection{Creating an Event} Teaching and learning activities are performed via events. A user can create an event for lecturing or learning.
 \subsubsection{Event Description} Since it is an social activity, event creator needs to write the details of the event.
 \subsubsection{Event Creator and Lecturers} Each event has a creator and lecturers. A lecturer can both a creator and lecturer of an event. 
 \subsubsection{Event Type} Event are classified against couple of parameters such as:
 \begin{itemize}
 \item \textbf{Indoor/Outdoor}. Where is it going to be held? Is it an indoor or outdoor activity?
   \item \textbf{Content of Event}. An indoor activity can be playing a game, having dinner etc.
   \item \textbf{Any Payment}. Does it include any sort of payment such as everyone pays its dinner?
     \item \textbf{Language Level of Audiance}. Level of audiance suggested to attend
 \end{itemize}
 \subsection{Event Notification System} Once an event is created or potential meet-up exists other users are notfied in a guided fashion.
 \subsubsection{Guided Concrete Notification for User} Once an event is created filtered notifications are sent according to the preferences of an user. For example, a available time, type of event, level of the event are used for filtering the notifications for a user.
 \subsubsection{Guided Ghost Notification for User} If common interests for type of events in common available time exists for couple of  people, a notfication is sent to create an event.
 \subsection{Confirmation System} 
 Once the event created set of users/a single user gets notification for the event and
 \begin{itemize}
 \item After getting notified for an event a user may send a request for being an lecturer or learner for this event.
 \item A creator can get buch of request for being lecturer/learner for the event. Creator builds the group/single user among the requests.
   \item People who are confirmed by creator are notified for confirmation of thier attendance.
   \item Limits of the group (number of lecturers and learners) are specified.
   \end{itemize}
 \subsection{Review System}
 Once the event is over, users can review via using two methods.
 \subsubsection{Commenting System} Commenting system can provide a way of reviewing the issues which are not quantified/qualified with teaching/learning activities. For example, an attendee may not be polite to other attendees and this may be mentioned in a comment. 
 \subsubsection{Ranking System}
  \begin{itemize}
 \item \textbf{Event}. Event creator can be review by other users.
 \item \textbf{Lecturer}. Lecturers can be reviwed by other users.
  \end{itemize}
........... Ismail : doc is filled up to this point .
  
\section{Development Infrastructure}
\subsection{Programming Languages}
\subsubsection{Front-End} Java-Script, PHP and HTML. Mike what do you think about front-end? Do you have suggestions ?
\subsubsection{Backend} Java
 \subsection{Integrated Development Environment} Eclipse, Netbeans etc. I am going to use Emacs :)
 \subsection{Testing Infrastructure} I used many mocking unit test environments we can use one of them for example mockito. We can write invariants for the system, pre-post conditions for methods etc.  
\section{Architectural Infrastructure} Priyasmita is going handle this part?
 \subsection{Data Base System}
 \subsection{Web Server}
\section{Maintainability}
\end{document}
