%%%%%%%%%%%%%%%%%%%%%%%%%%%%%%%%%%%%%%%%%
% Stylish Colored Title Page 
% LaTeX Template
% Version 1.0 (27/12/12)
%
% This template has been downloaded from:
% http://www.LaTeXTemplates.com
%
% Original author:
% Peter Wilson (herries.press@earthlink.net)
%
% License:
% CC BY-NC-SA 3.0 (http://creativecommons.org/licenses/by-nc-sa/3.0/)
% 
% Instructions for using this template:
% This title page compiles as is. If you wish to include this title page in 
% another document, you will need to copy everything before 
% \begin{document} into the preamble of your document. The title page is
% then included using \titleBC within your document.
%
%%%%%%%%%%%%%%%%%%%%%%%%%%%%%%%%%%%%%%%%%

%----------------------------------------------------------------------------------------
%	PACKAGES AND OTHER DOCUMENT CONFIGURATIONS
%----------------------------------------------------------------------------------------

\documentclass{article}

\usepackage[svgnames]{xcolor} % Required to specify font color

\newcommand*{\plogo}{\fbox{Drexel University}} % Generic publisher logo

\usepackage{graphicx} % Required for box manipulation

%----------------------------------------------------------------------------------------
%	TITLE PAGE
%----------------------------------------------------------------------------------------

\newcommand*{\rotrt}[1]{\rotatebox{90}{#1}} % Command to rotate right 90 degrees
\newcommand*{\rotlft}[1]{\rotatebox{-90}{#1}} % Command to rotate left 90 degrees

\newcommand*{\titleBC}{\begingroup % Create the command for including the title page in the document
\centering % Center all text

\def\CP{\textit{\Huge Meet Me For Language}} % Title

\settowidth{\unitlength}{\CP} % Set the width of the curly brackets to the width of the title
{\color{LightGoldenrod}\resizebox*{\unitlength}{\baselineskip}{\rotrt{$\}$}}} \\[\baselineskip] % Print top curly bracket
\textcolor{Sienna}{\CP} \\[\baselineskip] % Print title
{\color{RosyBrown}\Large Profiles for Socializing via Language Learning} \\ % Tagline or further description
{\color{LightGoldenrod}\resizebox*{\unitlength}{\baselineskip}{\rotlft{$\}$}}} % Print bottom curly bracket

\vfill % Whitespace between the title and the author name

{\Large\textbf{Priyasmita Bagchi}}\\ % Author name
{\Large\textbf{Bahareh Shakibajahromi}}\\ % Author name
{\Large\textbf{Mike Bok}}\\ % Author name
{\Large\textbf{Ismail Kuru}}\\ % Author name

\vfill % Whitespace between the author name and the publisher logo

\plogo\\[0.5\baselineskip] % Publisher logo
2016 % Year published

\endgroup}

%----------------------------------------------------------------------------------------
%	BLANK DOCUMENT
%----------------------------------------------------------------------------------------

\usepackage{longtable}
\usepackage{float}
\usepackage[section]{placeins}
\usepackage[toc,page]{appendix}
\usepackage{graphicx}
\usepackage[export]{adjustbox}
\usepackage{pdfpages}
\usepackage{amssymb}
\usepackage[utf8x]{inputenc}
\usepackage{amsmath}
\usepackage{amsthm}
\usepackage{listings}
\usepackage{stmaryrd}
\usepackage{graphics,graphicx}
%\usepackage[colorinlistoftodos]{todonotes}
\usepackage{import}
\usepackage{mathpartir}
\usepackage{aliascnt}
\usepackage{array}
\usepackage{comment}
\usepackage{indentfirst}
\usepackage{tikz}
\usetikzlibrary{matrix}
\usepackage{tikz}
\usepackage{environ}
\usetikzlibrary{calc, shapes, backgrounds}
\usepackage{verbatim}
\usepackage{pgf}
\usepackage{tikz}
\usepackage{tikz-qtree}
\usepackage{pgfplots}
\usepackage{caption,capt-of}
\usepackage{subcaption}
\usepackage{color}
\usepackage{xcolor}
\newtheorem{theorem}{Theorem}
\newtheorem{corollary}{Corollary}[theorem]
\newtheorem{Lemma}[theorem]{Lemma}
\theoremstyle{definition}
\newtheorem{definition}{Definition}[section]
\theoremstyle{remark}
\newtheorem*{remark}{Remark}
\usepackage{color}
\newcommand{\hilight}[1]{\colorbox{yellow}{#1}}
\usepackage{multicol}
\makeatletter

\usepackage{tcolorbox}
\usepackage{mdframed}
%\input{colin_macros.tex}

\mdfdefinestyle{MyFrame}{%
    linecolor=blue,
    outerlinewidth=2pt,
    roundcorner=20pt,
    innertopmargin=\baselineskip,
    innerbottommargin=\baselineskip,
    innerrightmargin=20pt,
    innerleftmargin=20pt,
    backgroundcolor=gray!50!white}

\pgfplotsset{}
\usetikzlibrary{shapes.geometric,arrows,fit,matrix,positioning}
\tikzset
{
    treenode/.style = {circle, draw=black, align=center, minimum size=0.1cm},
    subtree/.style  = {isosceles triangle, draw=black, align=center, minimum height=0.5cm, minimum width=0.5cm, shape border rotate=90, anchor=north}
}

\usetikzlibrary{graphs} % LaTeX
\usetikzlibrary{arrows,automata}
\usepackage{hyperref}

\definecolor{sh_comment}{rgb}{0.12, 0.38, 0.18 } %adjusted, in Eclipse: {0.25, 0.42, 0.30 } = #3F6A4D
\definecolor{sh_keyword}{rgb}{0.37, 0.08, 0.25}  % #5F1441
\definecolor{sh_string}{rgb}{0.06, 0.10, 0.98} % #101AF9


\def\lstsmallmath{\leavevmode\ifmmode \scriptstyle \else  \fi}
\def\lstsmallmathend{\leavevmode\ifmmode  \else  \fi}

\lstset {
 rulesepcolor=\color{black},
 showspaces=false,showtabs=false,tabsize=2,
 captionpos=b,
 xleftmargin=0.7cm, xrightmargin=0.5cm,
 lineskip=-0.3em,
 linewidth=\linewidth,
 language=C++,
 basicstyle=\ttfamily,
 keywordstyle=\color{blue}\ttfamily,
 stringstyle=\color{red}\ttfamily,
 commentstyle=\color{sh_comment}\ttfamily,
 morecomment=[l][\color{magenta}]{\#},
 escapeinside={<@}{@>}
}
\usepackage{listings}
\usepackage{color}
 
\definecolor{codegreen}{rgb}{0,0.6,0}
\definecolor{codegray}{rgb}{0.5,0.5,0.5}
\definecolor{codepurple}{rgb}{0.58,0,0.82}
\definecolor{backcolour}{rgb}{0.95,0.95,0.92}
 
\lstdefinestyle{mystyle}{
 %   backgroundcolor=\color{backcolour},   
    commentstyle=\color{codegreen},
    keywordstyle=\color{magenta},
    numberstyle=\tiny\color{codegray},
    stringstyle=\color{codepurple},
    basicstyle=\footnotesize,
    linewidth=\linewidth,
    breakatwhitespace=false,         
    breaklines=true,                 
    captionpos=b,                    
    keepspaces=true,                 
    numbers=left,                    
    numbersep=5pt,                  
    showspaces=false,                
    showstringspaces=false,
    showtabs=false,                  
    tabsize=2
}
 
\lstset{style=mystyle}
\usepackage{amsmath}

\begin{document}
\pagestyle{empty} % Removes page numbers

\titleBC % This command includes the title page
\newpage
\section{Motivation}
This project aims providing a platform for users to learn and improve their non-native languages via socializing with other  people. It is a profile based platform which enables users to create their profiles and get matched to a language learning event. This platform provides not only a pure learning activity but also getting socialized via meet-ups of users in created events.
\section{Functionality}
\subsection{Creating a Profile}
 \subsubsection{Qualification Aspect}
 \subsubsection{Lecturing Languages In Native}
 \subsubsection{Learning Languages}
 \subsection{Type of Lecturing Method}
 \subsection{Type of Learning Method}
\subsection{Creating an Event}
\subsubsection{Confirmation System} 
\subsection{Event Notification System}
 \subsubsection{Guided Concrete Notification for User}
 \subsubsection{Guided Ghost Notification for User}
\subsection{Review System}
  \subsubsection{Commenting System}
  \subsubsection{Ranking System}
\section{Development Infrastructure}
 \subsection{Programming Languages}
 \subsection{Integrated Development Environment}
 \subsection{Testing Infrastructure}
\section{Architectural Infrastructure}
 \subsection{Data Base System}
 \subsection{Web Server}
\section{Maintainability}
\end{document}
